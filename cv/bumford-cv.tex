\documentclass[12pt]{article}

%%%%% tables and lists %%%%%
\usepackage{array} % more control over column styles
\usepackage{longtable} % break table across page
  \setlength{\LTpre}{0pt}
  % \setlength{\LTpost}{0pt}
  \renewcommand{\arraystretch}{1.25} % row spacing
\usepackage{linegoal} % find length to end of line
\usepackage{enumitem}
  \setlist[itemize]{label=,leftmargin=\parindent,itemsep=0.25em}

%%% array columns without spaces
\newcolumntype{L}{@{}p{0.11\textwidth}@{}}
\newcolumntype{R}{@{}p{0.89\textwidth}@{}}
% \newcolumntype{C}{@{}c@{}}

%%% titles
% \usepackage{sectsty} % customize section title appearance
%   \allsectionsfont{\normalsize}
\usepackage[compact]{titlesec}
  \titleformat*{\section}{\small\bfseries\MakeTextUppercase}
  % \titleformat*{\subsection}{\color{gray}}
  \titleformat*{\subsection}{\bfseries}
\usepackage{textcase}
\usepackage{needspace}


%%%%% layout and formatting %%%%%

%%% general margins
\usepackage[margin=1in]{geometry}
\setlength{\parindent}{0pt}
% \setlength{\parskip}{0pt}

%%% font preferences
\usepackage{iftex} % conditional on compilation engine
\usepackage[LGR,T1]{fontenc}
\ifPDFTeX%
  \usepackage[utf8]{inputenc}
\else
  \usepackage[no-math]{fontspec}
  \DeclareTextCommand{\nobreakspace}{T1}{\leavevmode\nobreak\ }
\fi
\usepackage[varqu,varl]{zi4} % inconsolata as typewriter
\usepackage[type1]{biolinum} % biolinum as sans-serif
\usepackage{cochineal} % cochineal (crimson) for roman
% \let\digamma\relax
% \usepackage[%
%   scaled,
%   stdmathdigits=true,
%   stdmathitalics=true
% ]{lucimatx}  
\usepackage{substitutefont}
\usepackage[greek,english]{babel}
\substitutefont{LGR}{\rmdefault}{Cochineal-LF}
\usepackage[cochineal,bigdelims]{newtxmath}
\usepackage[cal=cm,scr=boondoxo,frak=lucida,bb=lucida]{mathalfa}
\useosf%
\useproportional%
\usepackage{microtype}

%%% tikz
\usepackage{tikz}
\usetikzlibrary{tikzmark,calc}

%%% links
\RequirePackage{xcolor}
\definecolor{splinkcolor}{RGB}{50, 132, 191}
\RequirePackage[colorlinks,breaklinks,
                linkcolor=splinkcolor, 
                urlcolor=splinkcolor, 
                citecolor=splinkcolor,
                filecolor=splinkcolor,
                plainpages=false,
                pdfpagelabels,
                bookmarks=false,
                pdfstartview=FitH]{hyperref}
\newcommand{\doi}[1]{\url{http://dx.doi.org/#1}}
\urlstyle{rm}

%%% macros
\newcommand{\with}{\&}
% \newcommand{\refmark}{\textcolor{gray}{\textgreek{r}}}
% \newcommand{\refmark}{\textcolor{gray}{\textdagger}}
\newcommand{\refmark}{\textcolor{gray}{}}


\begin{document}

\textbf{Dylan Bumford}\hfill
\textcolor{gray}{[last updated: \today]}

\bigskip

\begin{minipage}[t]{0.55\textwidth}
UCLA, Department of Linguistics\\
3125 Campbell Hall, Los Angeles, CA 90095
\\
web: \href{https://dylanbumford.com}{dylanbumford.com}\\
email: \href{mailto:dbumford@ucla.edu}{dbumford@ucla.edu}
\end{minipage}
\begin{minipage}[t]{\linegoal}
\raggedleft%
% web: \href{https://dylanbumford.com}{dylanbumford.com}\\
% email: \href{mailto:dbumford@ucla.edu}{dbumford@ucla.edu}
\end{minipage}

\bigskip
\bigskip

\section*{Employment}

\begin{longtable}{LR}
  2017--    & Assistant Professor\newline
              University of California, Los Angeles, Department of Linguistics
\end{longtable}

\medskip

\section*{Education}

\begin{longtable}{LR}
  2017        & Ph.D.~Linguistics, New York University\newline
                Dissertation: Split-scope effects in definite descriptions\newline
                Chair: Chris Barker
                % Committee: Anna Szabolcsi, Adrian Brasoveanu, Simon Charlow, Philippe Schlenker\\
                \\
  2014--15    & Visiting Researcher, Stanford University\\
  2010        & B.S.~Psychology, University of Texas at Austin\\
  2010        & B.A.~Mathematics, Linguistics, and Plan II Honors, University of
                Texas at Austin
\end{longtable}

\medskip

\section*{Publications% \
%  \NoCaseChange{\normalsize\mdseries\color{gray} (\refmark\ $=$\
%    journal article)}
}

\begingroup
\renewcommand{\arraystretch}{1.5} % more space between rows
\begin{longtable}{LR}
  %
  \begin{tikzpicture}[overlay,remember picture]%
  \node at ($({pic cs:p8}) +(-0.3,0.1)$) {\refmark};%
  \node at ($({pic cs:p7}) +(-0.3,0.1)$) {\refmark};%
  \node at ($({pic cs:p6}) +(-0.3,0.1)$) {\refmark};%
  \node at ($({pic cs:p5}) +(-0.3,0.1)$) {\refmark};%
  \end{tikzpicture}%
  %
  % \tikzmark{p4}%
  % In prep.  & Not the only game in town: Relative and indeterminate exclusive
  %             descriptions.\newline
  %             \textit{Journal of Semantics}, under revision.\\
  %
  % \begin{tikzpicture}[overlay,remember picture]
  % \node at ($({pic cs:p4}) +(-0.3,0.1)$) {\refmark};
  % \end{tikzpicture}
  %
  \tikzmark{p8}%
  ---   & Effect-driven interpretation.\newline
          With Simon Charlow.\newline
          Under contract for \textit{Cambridge University Press}.\\
  2022  & Polymorphic distributivity.\newline
          \textit{Natural Language Semantics} 30(3), 239--268.\newline
          \doi{10.1007/s11050-022-09195-5}\\
  \tikzmark{p7}%
  2022  & Composition under distributive natural transformations: Or, When
          Predicate Abstraction is impossible.\newline
          \textit{Journal of Logic, Language, and Information} 31(3), 287--307.\newline
          \doi{10.1007/s10849-022-09361-2}\\
  \tikzmark{p6}%
  2022  & Negative Polarity Items in definite superlatives.\newline
          With Yael Sharvit.\newline
          \textit{Linguistic Inquiry} 53(2), 255--293.\newline
          \doi{10.1162/ling_a_00409}\\
  2021  & Rationalizing Evaluativity.\newline
          With Jessica Rett.\newline
          \textit{Sinn und Bedeutung 25}, 187--204.\newline
          \doi{10.18148/sub/2021.v25i0.931}\\
  \tikzmark{p5}%
  2019  & Donkeys under discussion.\newline
          With Lucas Champollion and Robert Henderson.\newline
          \textit{Semantics \& Pragmatics} 12(1), 1--50.\newline
          \doi{10.3765/sp.12.1}\\
  %
  2018  & Binding into superlative descriptions.\newline
          \textit{Semantics and Linguistic Theory 28}, 325--344.\newline
          \doi{10.3765/salt.v28i0.4412}\\
  %
  2017  & Parsing with Dynamic Continuized CCG.\newline
          With Michael White, Jordan Needle, and Simon Charlow.\newline
          \textit{13th International Workshop on Tree-Adjoining Grammar and
          Related Formalisms}, 71--83.\newline
          \url{http://www.aclweb.org/anthology/W17-6208}\\
  %
  \tikzmark{p3}%
  2017  & Split-scope definites: Relative superlatives and Haddock
          descriptions.\newline
          \textit{Linguistics \& Philosophy} 40(6), 549--93.\newline
          \doi{10.1007/s10988-017-9210-2}\\
  %
  \tikzmark{p2}%
  2015  & Incremental quantification and the dynamics of pair-list phenomena.\newline
          \textit{Semantics \& Pragmatics} 8(9), 1--70.\newline
          \doi{10.3765/sp.8.9}\\
  %
  2015  & A cross-linguistic study of the non-at-issueness of exhaustive
          inferences.\newline
          With Emilie Destruel, Dan Velleman, Edgar Onea, Jingyang Xue, and
          David Beaver.\newline
          In Florian Schwarz (ed.) \textit{Experimental Perspectives on
          Presuppositions}. Springer, Dordrecht.\newline
          \doi{10.1007/978-3-319-07980-6_6}\\ % chktex 8
  %
  2014  & Universal quantification as iterated dynamic conjunction.\newline
          \textit{19th Amsterdam Colloquium}, 67--74.\newline
          \url{http://www.illc.uva.nl/AC/AC2013/uploaded_files/inlineitem/09_Bumford.pdf}\\
  %
  \tikzmark{p1}%
  2013  & Association with distributivity and the problem of multiple
          licensors for singular \textit{different}.\newline
          With Chris Barker.\newline
          \textit{Linguistics \& Philosophy} 36(5), 355--369.\newline
          \doi{10.1007/s10988-013-9139-z}\\ % chktex 8
  %
  2012  & It-clefts are IT (inquiry-terminating) constructions.\newline
          With Dan Velleman, David Beaver, Emilie Destruel, Edgar Onea, and
          Elizabeth Coppock.\newline
          \textit{Semantics and Linguistic Theory 22}, 441--460.\newline
          \doi{10.3765/salt.v22i0.2640}
  %
  \begin{tikzpicture}[overlay,remember picture]%
  \node at ($({pic cs:p3}) +(-0.3,0.1)$) {\refmark};%
  \node at ($({pic cs:p2}) +(-0.3,0.1)$) {\refmark};%
  \node at ($({pic cs:p1}) +(-0.3,0.1)$) {\refmark};%
  \end{tikzpicture}%
\end{longtable}
\endgroup

\medskip

\section*{Presentations}

\begin{longtable}{LR}
  2021 & Rationalizing evaluativity.
         \textit{Sinn und Bedeutung}.\\
  %
  2020 & Superlative updates.
         Invited talk, NYU.\\
  %
  2019 & Superlative scope, comparison classes, and negative polarity.
         Invited colloquium, USC.\\
  %
  2019 & Superlative scope, comparison classes, and negative polarity.
         Invited colloquium, UC Irvine.\\
  %
  2019 & Superlative scope, comparison classes, and negative polarity.
         Invited colloquium, University of Arizona.\\
  %
  2018 & Effectful composition in natural language.
         With Simon Charlow.
         Invited talk, \textit{North American Summer School of Logic, Language,
         and Information}, CMU.\\
  %
  2018 & Binding into superlative descriptions.
         \textit{Semantics and Linguistic Theory 28}, MIT.\\
  %
  2017 & Proofs as programs (and programs as proofs).
         Invited lecture, ``Computational Semantics''
         graduate seminar (Simon Charlow, instructor), Rutgers University.\\
  % 
  2017 & Parsing with Dynamic Continuized CCG.
         With Michael White, Jordan Needle, and Simon Charlow.
         \textit{13th International Workshop on Tree-Adjoining Grammar and
         Related Formalisms}, Ume\aa.\\
  %
  2017 & Split-scope effects in definite descriptions.
         Invited colloquium, Boston University.\\
  %
  2017 & Split-scope effects in definite descriptions.
         Invited colloquium, UCLA.\\
  %
  2016 & Polyadicity in descriptions.
         Invited talk, University of Pennsylvania.\\
  %
  2016 & Monadic dynamic semantics: Side effects and scope.
         With Simon Charlow.
         \textit{Natural Language and Computer Science 4}, Columbia University.\\
  %
  2016 & The rabbit in the hat is back: Definites with joint uniqueness
         presuppositions.
         With Lucas Champollion and Linmin Zhang.
         \textit{Definiteness Across Languages}, UNAM.\\
  %
  2016 & Decomposing definiteness: Effects of delayed quantification in
         descriptions.
         \textit{Semantics and Linguistic Theory 26}, UT Austin.\\
  %
  2016 & Split-scope definites.
         Invited talk, Queen Mary University.\\
  %
  2015 & Temporal modification without pronouns.
         Invited talk, NYU Continuations Workshop.\\
  % 
  2015 & Adjectives of comparison.
         With Chris Barker.
         Invited talk, UCSC S-Circle.\\
  % 
  2014 & Incremental quantification and the dynamics of pair-list phenomena.
         Invited talk, Stanford Construction of Meaning Workshop.\\
  % 
  2013 & Universal quantification as iterated dynamic conjunction.
         \textit{19th Amsterdam Colloquium}. \\
  % 
  2013 & Generalized association with distributivity.
         With Chris Barker.
         \textit{Semantics and Linguistic Theory 23}, UCSC.\\
  %
  2012 & It-clefts are IT (inquiry-terminating) constructions.
         With Dan Velleman, David Beaver, Emilie Destruel, Edgar Onea, and
         Elizabeth Coppock.
         \textit{Semantics and Linguistic Theory 22}, University of Chicago.\\
  % 
  2011 & ``Yes, but\dots'' --- Exhaustivity and at-issueness across languages.
          With Dan Velleman, David Beaver, Emilie Destruel, and Edgar Onea.
          \textit{Workshop on Projection, Entailment, Presupposition, and
          Assertion}.\\
  % 
  2009 & Effects of grammatical form and familiarity on metaphor
         comprehension.
         With Lauretta Reeves.
         \textit{Metaphor Festival}, Stockholm.
\end{longtable}

\medskip

\needspace{5\baselineskip}
\section*{Teaching}

\subsection*{Primary instructor at UCLA}

\begin{longtable}{LR}
  2022 & \textlf{Ling 252}: English as a programming language (graduate seminar)\\
  %
  2021 & \textlf{Ling 200C}: Semantic Theory I (graduate)\\
  %
  2021 & \textlf{Ling 120C}: Semantics I (undergraduate)\\
  %
  2021 & \textlf{Ling 252}: New issues for the semantics of questions (graduate seminar)\\
  %
  2020 & \textlf{Ling 200C}: Semantic Theory I (graduate)\\
  %
  2020 & \textlf{Ling 222}: Semantic Theory III, Dynamic Semantics (graduate)\\
  %
  2020 & \textlf{Ling 120C}: Semantics I (undergraduate)\\
  %
  2019 & \textlf{Ling 200C}: Semantic Theory I (graduate)\\
  %
  2019 & \textlf{Ling 165C}: Semantics II (undergraduate)\\
  %
  2019 & \textlf{Ling 252}: Relative Clauses (graduate seminar)\\
  %
  2018 & \textlf{Ling 200C}: Semantic Theory I (graduate)\\
  %
  2018 & \textlf{Ling 180}: Mathematical Structures in Language (undergraduate)\\
  %
  2018 & \textlf{Ling 252}: Definiteness and Degrees (graduate seminar)\\
  %
  2018 & \textlf{Ling 165C}: Semantic Theory II (graduate)\\
  %
  2018 & \textlf{Ling 201C}: Semantics II (undergraduate)\\
  %
  2017 & \textlf{Ling 20}: Introduction to Linguistic Analysis (undergraduate)
  % 
\end{longtable}

\subsection*{Primary instructor elsewhere}%

\begin{longtable}{LR}
  2022 & Effectful composition in natural language semantics (European Summer
         School in Logic, Language, and Information), co-taught with Simon
         Charlow\\
  %
  2015 & Monads and Natural Language (European Summer School in Logic, Language,
         and Information), co-taught with Chris Barker
\end{longtable}

\subsection*{Teaching assistant}

\begin{longtable}{LR}
  2016 & Introduction to Semantics (NYU, Chris Barker)
         %35 students\newline
         %\hspace*{0.5cm}\textendash\
         %7 primary lectures, weekly recitation sections, assignment and exam
         %development
         \\
  %
  2015 & Introduction to Semantics (NYU, Chris Barker)
         %30 students
         \\
  %
  2013 & Language and Mind (NYU, Anna Szabolcsi and Brian McElree)
         %75 students
         \\
  %
  2013 & Grammatical Analysis (NYU, Mark Baltin)
         % 25 students
         \\
  %
  2009 & Mind and Reason (UT, David Beaver)
         %20 students
         \\
  %
  2008 & Punishment in Society (UT, Robert Pitman)
         %10 students
\end{longtable}


\medskip

\section*{Advising}

\subsection*{Faculty mentor}

\begin{longtable}{LR}
  2022    & Laurel Perkins, UCLA Linguistics
\end{longtable}

\subsection*{PhD dissertation chair}

\begin{longtable}{LR}
  2020    & Maayan Abenina-Adar, \textit{Expressing ignorance with determiner
            phrases}, UCLA Linguistics
\end{longtable}

\subsection*{MA thesis chair}

\begin{longtable}{LR}
  2022    & Kalen Chang, \textit{Restrictiveness and the scope of adjectives},
            UCLA Linguistics
\end{longtable}

\subsection*{Committee member}

\begin{longtable}{LR}
  ongoing & Huilei Wang, PhD Dissertation\\
  %
  ongoing & Yang Wang, PhD Dissertation\\
  %
  2022    & Bethany Sturman, \textit{The semantics of quotation intonation},
            PhD Dissertation\\
  %
  2022    & Maura O'Leary, \textit{The Evaluation Times of Nominal Predicates},
            PhD Dissertation\\
  %
  % inc.    & Mia Teixeira, \textit{Learning scalar items}, PhD Dissertation\\
  %
  2021    & Deborah Wong, \textit{Wh-in situ sluicing in Minimalist Grammars},
            PhD Dissertation\\
  %
  2021    & Yang Wang, \textit{Regular languages extended with reduplication:
            Formal models, proofs and illustrations}, MA Thesis\\
  %
  2020    & Iara Mantenuto, \textit{Alternate Particles in San Sebasti\'{a}n del
            Monte Mixtec and Beyond}, PhD Dissertation\\
  %
  2020    & Richard Stockwell, \textit{The contrast condition on ellipsis}, UCLA
            PhD Dissertation\\
  %
  2020    & Mia Teixeira, \textit{Against Exclusion-Based Counterfactuality},
            MA Thesis\\
  %
  2020    & Phillip Barnett, \textit{Arc-Eager Effects on Probabilistic
            Left-Corner Language Models}, MA Thesis\\
  %
  2019    & Madeleine Booth, \textit{Reconstruction and Resumptive Pronouns in
            Cairene Arabic}, MA Thesis
\end{longtable}

\subsection*{Research project supervisor}

\begin{longtable}{LR}
  2022     & Kalen Chang, \textit{Nonrestrictive Adjective Modification and
             Not-At-Issue Meaning}, Graduate Student Research Mentorship
\end{longtable}

\subsection*{Other advising}

\begin{longtable}{LR}
  2022     & Pop-up mentor, North American Summer School of Logic, Language, and
             Information
  \\
  2020     & Florence Verity, \textit{Implicatures in continuation-based dynamic
             semantics by reasoning in the context}, External examiner, ANU MPhil
             thesis
  \\
  2019--20 & Isabel Zhou, CollegeBoard AP Research Course, Expert Advisor
\end{longtable}

\medskip

\section*{Honors, Grants, and Awards}

\begin{longtable}{LR}
  2018     & Dean's Discretionary Fund, \$1,000\\
  2013--16 & National Science Foundation Graduate Research
             Fellowship\newline
             \hspace*{0.5cm}\textendash\
             3-year grant covering tuition, research expenses, and living
             stipend\\
  %
  2013     & NYU Curricular Development Challenge Fund (with Lucas Champollion, PI)\newline
             \hspace*{0.5cm}\textendash\
             \$4,800 grant to develop pedagogical software for semantics courses\\
  %
  2011--16 & Henry M. MacCracken Fellowship\newline
             \hspace*{0.5cm}\textendash\
             5-year funding package from NYU, including tuition and stipend\\
  % 
  2011     & George H. Mitchell Grand Prize for Academic Research\newline
             \hspace*{0.5cm}\textendash\
             \$20,000 prize awarded to one UT undergraduate per year for
             thesis research
\end{longtable}


\medskip

\needspace{5\baselineskip}
\section*{Service and Experience}

\subsection*{Journal reviewing}

\begin{longtable}{LR}
  ongoing &
    \textit{Glossa}% (2022)
    ,
    \textit{Linguistic Inquiry}% (2021)
    ,
    \textit{Language}% (2019)
    ,
    \textit{Synthese}% (2018)
    ,
    \textit{Semantics \& Pragmatics}% (2012, 2017--22)
    ,
    \textit{Linguistics \& Philosophy}% (2017--19)
    ,
    \textit{Natural Language Semantics}% (2017)
    ,
    \textit{Journal of Semantics}% (2016--19)
    ,
    \textit{Journal of Language Modelling}% (2017) 
\end{longtable}

\subsection*{Conference reviewing}

\begin{longtable}{LR}
  ongoing &
    \textit{Computational and Experimental Explanations in Semantics and
             Pragmatics}% (2022)
    ,
    \textit{Semantics and Philosophy in Europe}% (2019)
    ,
    \textit{North East Linguistic Society}% (2019)
    ,
    \textit{West Coast Conference on Formal Linguistics}% (2018--22)
    ,
    \textit{Semantics and Linguistic Theory}% (2018, 20--22)
    ,
    \textit{Sinn und Bedeutung}% (2018--19)
    ,
    \textit{Logic and Engineering of Natural Language Semantics}% (2017)
    ,
    \textit{Amsterdam Colloquium}% (2020--)
\end{longtable}

\subsection*{Professional service}

\begin{longtable}{LR}
  2022--     & Technology Coordinator, \textit{Semantics and Linguistic Theory}\\
  2022--22   & Co-organizer, \textit{California Universities Semantics and
               Pragmatics 13}\\
  2019--     & Steering Committee, \textit{Semantics and Linguistic Theory}\\
  2018--19   & Co-organizer, \textit{Semantics and Linguistic Theory 29}\\
  2017--18   & Program Committee, \textit{West Coast Conference on Formal
               Linguistics 36}\\
  2013--14   & Program Committee, \textit{Semantics and Linguistic Theory 24}
\end{longtable}

\subsection*{Departmental service}

\begin{longtable}{LR}
  2021--21   & Climate Survey Committee, UCLA Linguistics\\
  2021--21   & Iranian Post-Doc Search Committee, UCLA Linguistics\\
  2020--21   & Visiting Students and Scholars Committee, UCLA Linguistics\\
  2019--20   & Computational Linguistics Job Search Committee, UCLA Linguistics\\
  2019--20   & Merit Review Committee, UCLA Linguistics\\
  2018--21   & Linguistics and Computer Science Faculty Advisor, UCLA Linguistics\\
  2018--20   & Website Committee, UCLA Linguistics\\
  2013--16   & Organizer, NYU Semantics Group
\end{longtable}

\subsection*{Other employment}
\begin{longtable}{LR}
  2013--15   & Software development, The Lambda Calculator
               %\hspace*{0.5cm}\textendash\
               %Pedagogical tool to assist in teaching the lambda calculus
               \\
  2012       & Research Assistant, Stephanie Harves\\
  2009--11   & Research Assistant, David Beaver
\end{longtable}

% \section*{Skills}

% \begin{itemize}[itemsep=0pt]
% \item
%   Haskell, Java, Python
% \item
%   Html, CSS, Javascript
% \item
%   R, Matlab, Mechanical Turk
% \end{itemize}

\end{document}
